Use the project website

\url{https://rivet.hepforge.org/trac/wiki/GettingStarted}

Installation with afs:

The description below is based on a build from CERN's lxplus6 SLC6 machines.

1) Download the bootstrap-lcg script into a temporary working directory, and make it executable:
\begin{lstlisting}
  cd /scratch/rivet
  (cd $HOME)
  wget http://rivet.hepforge.org/hg/bootstrap/raw-file/2.1.1/rivet-bootstrap-lcg
  chmod +x rivet-bootstrap-lcg
\end{lstlisting}
  
  
or for Rivet 1.x:
\begin{lstlisting}
  cd /scratch/rivet
  wget http://rivet.hepforge.org/hg/bootstrap/raw-file/1.9.0/rivet-bootstrap-lcg
  chmod +x rivet-bootstrap-lcg
\end{lstlisting}


2) Check/edit the script. Look in the script to see its target setup and make edits if you need to: you may want to change the LCG tag, the compiler environment that is set up, whether LCG packages are to be used from AFS, and the install and build locations:
\begin{lstlisting}
  less rivet-bootstrap-lcg \#\# and read...
  nano rivet-bootstrap-lcg \#\# and edit...
\end{lstlisting}


3) Run the script. By default it will install to \lstinline|\$PWD/local|, where \lstinline|\$PWD| is the current directory. If you need to change that, edit the file as above.

\lstinline|./rivet-bootstrap-lcg|

We will refer to the installation root path as \$PREFIX.

(this will take a while)

If you have trouble with the Boost library (hopefully you won't) see TroubleshootingBoost.

Setting up the environment

After the script grinds away for a while, it will tell you that it is finished and how to set up a runtime environment (similar to that used inside the build script) for running Rivet. A sourceable rivetenv.(c)sh script is provided for (c)sh shell users to help set up this environment. Here's how to set up the environment and then test the rivet program's help feature and analysis listing:
\begin{lstlisting}
  source \$PREFIX/rivetenv.sh
  rivet --help
  rivet --list-analyses
\end{lstlisting}

If that works, everything is installed correctly. If you are using the bash shell in your terminal, then Rivet will offer you programmable tab completion: try typing rivet and pressing the Tab key!

You may wish to add the environment variable settings to your ~/.bashrc shell config file, so that Rivet will work without needing any special session setup.

You can now check out the FirstRivetRun guide.


run it: 

\lstinline|source ~/.bashrc|
\lstinline|setupRivet|

Rivet runs with HepMC files. To convert the lhe files generated by MadGraph download lhef2hepmc from the repo:

\lstinline|svn checkout http://rivet.hepforge.org/svn/contrib/lhef2hepmc/ /local_path/|

In the Makefile change the path to HepMC to:

\lstinline|HEPMC_PREFIX=/afs/cern.ch/sw/lcg/external/HepMC/2.06.06/x86_64-slc6-gcc48-opt|

This is where HepMC is installed on AFS.

Then run \lstinline|./lhef2hepmc --help|


