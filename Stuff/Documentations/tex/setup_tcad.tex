TCAD

get a license (ask Mathieu Benoit), agree by replying the mail

get a working space from account manager (\url{http://account.cern.ch}, max. 100\,GB)

copy some scripts 

login onto lxplus \lstinline|ssh lmeng@lxplus.cern.ch -Y|

go to working space

\lstinline|cd /afs/cern.ch/work/l/lmeng/|

\lstinline|source setupTCAD.sh| or \lstinline|. setupTCAD.sh|

\begin{lstlisting}
bla

\end{lstlisting}

to start \lstinline|swb &| for \emph{Sentaurus Workbench}

http://web.stanford.edu/class/ee328/swb/swb\_d.html for a tutorial

In Sentaurus Workbench, parameters can be defined and multiple values can be assigned to them to create splits in experiments (simulations). Each value of a parameter creates an additional experiment; therefore, for N values of a parameter, there are N experiments. For two parameters, P1 and P2, with M and N values, respectively, there are M x N possible experiments.

The creation of different scenarios is particularly helpful when many parameters are used. The resulting experiments can be classified into separate scenarios to represent different physical situations.

Extensions $\rightarrow$ Commandline prompt here

All interactive operations associated with running a project can be accomplished at a UNIX prompt. The commands for submitting jobs, preprocessing, and cleaning up project directories are 

\lstinline|gsub (gjob)| submitting jobs
\lstinline|spp| proprocessing
\lstinline|gcleanup|


For more information about the commands executed (at the prompt), refer to the Sentaurus Workbench User Guide or use the UNIX command:

> <command> -h[elp]
for example:

> gsub -h


Node navigation

 a colon (:) is the operator used to navigate between different nodes within the same tool of the project.
 a vertical bar (|) is the operator to navigate between different tool nodes in the project.




